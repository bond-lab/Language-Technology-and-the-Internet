\PassOptionsToPackage{xetex}{xcolor}
\PassOptionsToPackage{xetex}{graphicx}
\documentclass[25pt,a4paper,landscape,headrule,footrule,xetex]{foils}
\usepackage{listings}
%%
%%%  Macros
%%%
\newcommand{\logo}{~}
\MyLogo{HG2052 (2020)}
%\newcommand{\Story}{\SHA{HOUN}{The Hound of the Baskervilles}}

\newcommand{\header}[3]{%
\title{\vspace*{-2ex} \Large HG2052
\\\large  Language, Technology and the Internet
\\[2ex] \Large  \emp{#2}}
\author{\blu{Francis Bond}   \\ 
\normalsize  \textbf{Division of Linguistics and Multilingual Studies}\\
\normalsize  \url{http://www3.ntu.edu.sg/home/fcbond/}\\
\normalsize  \texttt{bond@ieee.org}}
\MyLogo{HG2052 (2020)}
\date{#1}
\renewcommand{\logo}{#2}
 \hypersetup{
   pdfinfo={
     Author={Francis Bond},
     Title={#1: #2},
     Subject={HG2052: Language, Technology and the Internet},
     Keywords={Language, Technology, Internet},
     License={CC BY 4.0}
   }
 %  pdfcopyright={Copyright © Francis Bond. Creative Commons 4.0 Attribution License.}
 %  pdflicenseurl={http://creativecommons.org/licenses/by/4.0/}
 }
}


%%
%% Multilingual Stuff
%%
\usepackage[a4paper,landscape,margin=25mm]{geometry}

\usepackage{fontenc}
\usepackage{polyglossia}
\setmainlanguage{english}
\setmainfont{TeX Gyre Pagella}
%\setmainfont{Linux Libertine}
%\setmainfont{Charis SIL}
\newfontfamily{\ipafont}{Gentium}
\newcommand{\ipa}[1]{{\ipafont\selectfont #1}}
\usepackage{xeCJK}

\setCJKmainfont{Noto Sans CJK SC}
\setCJKsansfont{Noto Sans CJK SC}
%\setCJKttfont{Noto Sans CJK SC}
%\setCJKmainfont{WenQuanYi Micro Hei}
%\clearpage
%\setCJKmainfont{AR PL SungtiL GB}

\usepackage[xetex]{xcolor}
\usepackage[xetex]{graphicx}
\newcommand{\blu}[1]{\textcolor{blue}{#1}}
\newcommand{\grn}[1]{\textcolor{green}{#1}}
\newcommand{\hide}[1]{\textcolor{white}{#1}}
\newcommand{\emp}[1]{\textcolor{red}{#1}}
\newcommand{\txx}[1]{\textbf{\textcolor{blue}{#1}}}
\newcommand{\lex}[1]{\textbf{\mtcitestyle{#1}}}

\usepackage{pifont}
\renewcommand{\labelitemi}{\textcolor{violet}{\ding{227}}}
\renewcommand{\labelitemii}{\textcolor{purple}{\ding{226}}}

\newcommand{\subhead}[1]{\noindent\textbf{#1}\\[5mm]}

\newcommand{\Bad}{\emp{\raisebox{0.15ex}{\ensuremath{\mathbf{\otimes}}}}}
\newcommand{\bad}{*}

\newcommand{\com}[1]{\hfill \textnormal{(\emp{#1})}}%
\newcommand{\cxm}[1]{\hfill \textnormal{(\txx{#1})}}%
\newcommand{\cmm}[1]{\hfill \textnormal{(#1)}}%
\usepackage{amssymb}
\usepackage{relsize,xspace}
\newcommand{\into}{\ensuremath{\rightarrow}\xspace}
\newcommand{\ent}{\ensuremath{\Rightarrow}\xspace}
\newcommand{\nent}{\ensuremath{\not\Rightarrow}\xspace}
\newcommand{\tot}{\ensuremath{\leftrightarrow}\xspace}
\usepackage{url}
\usepackage[hidelinks]{hyperref}
\hypersetup{
     colorlinks,
     linkcolor={blue!50!black},
     citecolor={red!50!black},
     urlcolor={blue!80!black}
}
%\usepackage{hyperxmp}
\usepackage{url}
\newcommand{\lurl}[1]{\MyLogo{\url{#1}}}

\usepackage{mygb4e}
\let\eachwordone=\itshape
\newcommand{\lx}[1]{\textbf{\textit{#1}}}
\newcommand{\ix}{\ex\it}

\newcommand{\cen}[2]{\multicolumn{#1}{c}{#2}}
%\usepackage{times}
%\usepackage{nttfoilhead}
\newcommand{\myslide}[1]{%
\foilhead[-25mm]{\raisebox{12mm}[0mm]{\emp{#1}}}%
\leftheader{}%
\MyLogo{\logo}}

\newcommand{\mytask}[1]{%
\foilhead[-25mm]{\raisebox{12mm}[0mm]{\emp{#1}}}
\leftheader{🔍 Hi}%
\MyLogo{\logo}}

\newcommand{\myslider}[1]{\rotatefoilhead[-25mm]{\raisebox{12mm}[0mm]{\emp{#1}}}}
%\newcommand{\myslider}[1]{\rotatefoilhead{\raisebox{-8mm}{\emp{#1}}}}

\newcommand{\section}[1]{\myslide{}{\begin{center}\Huge \emp{#1}\end{center}}}

\usepackage{tcolorbox}
% \newcommand{\task}{\marginpar{\raisebox{-1ex}{\large
%       \tcbox[colframe=red,colback=white,arc=3pt]{\textbf{?}}}}}
% \newcommand{\task}{\marginpar{\raisebox{-1ex}{
%       \hspace{-0.5em}\tcbox[colframe=red,colback=white,arc=3pt]{%
%         \includegraphics[width=1.5em]{pics/detective}}}}}
\newcommand{\task}{\marginpar{\raisebox{-2ex}{
      \hspace{-0.5em}\reflectbox{\includegraphics[width=2em]{pics/detective}}}}}

\usepackage[lyons,j,e,k]{mtg2e}
\renewcommand{\mtcitestyle}[1]{\textcolor{teal}{\textsl{#1}}}
%\renewcommand{\mtcitestyle}[1]{\textsl{#1}}
\newcommand{\chn}{\mtciteform}
\newcommand{\cmn}{\mtciteform}
\newcommand{\iz}[1]{\textup{\texttt{\textcolor{blue}{\textbf{#1}}}}}
\newcommand{\con}[1]{\textsc{#1}}
\newcommand{\gm}{\textsc}
\newcommand{\cmp}[1]{{[\textsc{#1}]}}
\newcommand{\sr}[1]{\ensuremath{\langle}#1\ensuremath{\rangle}}
\usepackage[normalem]{ulem}
\newcommand{\ul}{\uline}
\newcommand{\uul}{\uuline}
\newcommand{\wl}{\uwave}
\newcommand{\vs}{\ensuremath{\Leftrightarrow}~}
%%%
%%% Bibliography
%%%
\usepackage{natbib}
%\usepackage{url}
\usepackage{bibentry}


%%% From Tim
\newcommand{\WMngram}[1][]{$n$-gram#1\xspace}
\newcommand{\infers}{$\rightarrow$\xspace}



\usepackage{rtrees,qtree}
\renewcommand{\lf}[1]{\br{#1}{}}
\usepackage{avm}
%\avmoptions{topleft,center}
\newcommand{\ft}[1]{\textsc{#1}}
\newcommand{\val}[1]{\textit{#1}}
\newcommand{\typ}[1]{\textit{#1}}
\avmfont{\sc}
%\avmvalfont{\sc}
\renewcommand{\avmtreefont}{\sc}
\avmsortfont{\it}


%%% From CSLI book
\newcommand{\mc}{\multicolumn}
\newcommand{\HD}{\textbf{H}\xspace}
\newcommand{\el}{\< \>}
\makeatother
\long\def\smalltree#1{\leavevmode{\def\\{\cr\noalign{\vskip12pt}}%
\def\mc##1##2{\multispan{##1}{\hfil##2\hfil}}%
\tabskip=1em%
\hbox{\vtop{\halign{&\hfil##\hfil\cr
#1\crcr}}}}}
\makeatletter

\newcommand{\sh}[1]{\href{https://www.arthur-conan-doyle.com/index.php?title=#1}{#1}}
\newcommand{\SHA}[2]{\href{https://www.arthur-conan-doyle.com/index.php?title=#1}{\textit{#2}}}

\newcommand{\prompt}[1]{\fbox{
    \begin{minipage}{0.9\linewidth}
      #1
    \end{minipage}}}


\header{Lecture 11}{Large Language Models}


\begin{document}

\bibliographystyle{apalike}
\nobibliography{abb,mtg,nlp,ling}

\maketitle\makexeCJKinactive

%\input{schedule}


\section{LLM\\ Large Language Models}


\myslide{Large Language Models}

\begin{itemize}
\item Generative models that capture extended context in order to
  predict the next token (word, sub-word or symbol) 
\item The main innovation is in the sheer size
\item They model how language is used
\item They can appear know a lot about the world
  \begin{itemize}
  \item Humans naturally make meaning of language we encounter and     imagine the mind behind that language
     \begin{itemize}
  \item Artificial agents have limited capacity for communicative intent
  \item And some natural language systems have none
  \end{itemize}
\item Dealing with the risks of language
    technology requires recognizing and accounting for the above
 \\   \href{https://faculty.washington.edu/ebender/papers/Bender-HiTZ-2023.pdf}{Meaning
      making with artificial interlocutors and risks of language
      technology}, Talk at HiTz, Emily Bender 2023
  \end{itemize}
%\item We will look at ChatGPT 3.5, but there are many similar models
\end{itemize}

\myslide{Generative Pre-trained Transformer (GPT) models}

\begin{itemize}
\item Non-supervised --- just needs text (and/or code)
 \\[3ex]
  \begin{tabular}{clrrl}
    Date & Name & Parameters$^{*}$ & Data & Type \\ \hline
    2018 &  GPT &  117 &  4.5 GB &  Books \\
    2019 &  GPT-2 & 1,500  &  40 GB & Web \\
    2020 &  GPT-3 &  175,000 &  570 GB &
      Web 82\%, Books 16\%  Wiki 3\% \\ 
    2021 & ChatGPT$^{**}$ \\
    2023 & GPT-4 & {$^{***}$}1,750,000 & ? &  text and images
  \end{tabular}
  \\ {*} million
  \\ {**}  GPT-3.5 fine-tuned to follow instructions and interact
  \\ {***} estimate
\end{itemize}

Lambdalabs estimates a cost of around \$4.6 million US dollars and 355 years to train GPT-3 on a single GPU in 2020

\myslide{ChatGPT Intro}

\begin{itemize}
\item The base is generative pre-trained transformer (GPT) model
  \\ based on the transformer architecture developed by Google
\item Training data includes software manual pages, bulletin board systems, multiple programming languages, wikipedia, books and more (up to Jan 2022)
  \item Fine-tuned for conversational applications using a combination
  of supervised and reinforcement learning techniques from human feedback
\item Interact with a prompt, to which you get a response
  \begin{itemize}
  \item  keeps a context of 4k tokens ($\approx 500$ tokens/A4 page)
  \item  longer context (up to 32k) in newer models
  \end{itemize}
\item Free interface, beyond a certain amount you are charged per
  token
  \\ (from   $\$0.002--\$0.12$/token)        
  
\end{itemize}

\myslide{Prompt Engineering}

The first prompt provides a starting point or direction for ChatGPT. The response follows the same tone, context, and style as the prompt.

\begin{itemize}
\item Start simple - start with a simple prompt and build on it
\item Give a command - start the prompt with an action word like "Write", "Create", or "Summarize" instead of "Can you"
\item Add Context - add specific and relevant context to the task you want to perform
\item Add Expectations - add clear and direct expectations for the content, like how long it should be and what to include
\end{itemize}

\prompt{Create a list of interesting places to visit in Olomouc:\\
  - they should be cheap\\
  - they should be fun}

\myslide{Role Prompting}

A simple way to add context to a prompt is to assign a role to ChatGPT.

\prompt{You are a student.\\
Create a list of interesting places to visit in Olomouc:\\
  - they should be cheap\\
  - they should be fun}

\prompt{You are an artist.\\
Create a list of interesting places to visit in Olomouc:\\
  - they should be cheap\\
  - they should be fun}

\prompt{You are a foodie.\\
Create a list of interesting places to visit in Olomouc:\\
  - they should be cheap\\
  - they should be fun}

\myslide{You can add a lot more information}
\begin{itemize}
\item \txx{Zero-shot} prompting means providing a prompt with no examples
\prompt{Classify the text into positive, neutral or negative: \\
Text: That shot selection was awesome. \\
Classification: \\
}
\newpage

\item \txx{Few-shot} prompting means you give the system some examples





\prompt{Text: Today the weather is fantastic \\
Classification: Pos \\
Text: The furniture is small. \\
Classification: Neu \\
Text: I don't like your attitude \\
Classification: Neg \\
Text: That shot selection was awful \\
Classification:
\\}

\end{itemize}

\myslide{ChatGPT example with formatting}

  \fbox{\begin{minipage}{0.9\linewidth}
Write a short introduction to computational linguistics \\ using LaTeX with linguistic examples
  \end{minipage}}
\smaller
\begin{verbatim}
\documentclass{article}

\begin{document}

\section{Introduction to Computational Linguistics}

Computational Linguistics is an interdisciplinary field that 
combines knowledge and techniques from linguistics, computer 
science, artificial intelligence, and cognitive psychology to
understand and model language from a computational perspective.
\end{verbatim}

\begin{verbatim}
For example, computational linguistics can be used to study 
the structure of language at different levels, such as syntax 
and morphology. In syntax, computational linguistics can help 
identify the rules that govern how words are combined to form 
phrases and sentences. For instance, in the sentence "The cat 
chased the mouse", computational linguistics can analyze the 
syntactic structure to identify that "The cat" is the subject,
 "chased" is the verb, and "the mouse" is the object.

Similarly, in morphology, computational linguistics can help 
identify the rules that govern how words are formed from smaller
 units of meaning, called morphemes. For example, in the word
 "unhappiness", computational linguistics can analyze the
morphological structure to identify that "un-" is a prefix 
meaning "not", "-happi-" is the root meaning "happy", and 
"-ness" is a suffix meaning "state of".
\end{verbatim}

\begin{verbatim}
Overall, computational linguistics provides a powerful set of
tools for analyzing language and understanding its underlying
structure. It has many practical applications, such as natural
language processing, machine translation, and speech recognition,
and continues to be an active area of research and development.
\end{document}
\end{verbatim}

\myslide{Some possible uses}

\begin{itemize}
\item Brainstorming
\item First Draft
\item Find Title
\item Email
\item Summarize
\item Code
\item Debug
\end{itemize}

\myslide{Brainstorming}

\prompt{What are some unique ideas for activities for 10 five-year-old girls at a birthday party?}

\prompt{How can we combine ideas 4 and 6? How can we make this into a unique experience that the children will remember?}

\myslide{First Draft}

\prompt{Write an outline for a tutorial on prompt engineering}

\prompt{For a tutorial on prompt engineering, write a section about "Understanding the Basics of Prompts"}



\myslide{Find Title}

\prompt{Suggest 10 titles for a short story about a robot posing as a student. It is written from a teenager's perspective, and the robot is almost a perfect copy of a human. They find friends and fall in love. But they still can't solve captchas when browsing the internet.}

\prompt{Number 8 suits the story well, as the story is mainly about the robot figuring out how to stay true to its robotic nature but at the same time develop as a human. Based on that, suggest 10 more titles.}


\myslide{Email}

\prompt{Write a professional email requesting the day off on the 17th of November? This is a holiday in the Czech Republic and though we understand that this is not usually a day off in the USA, we want to respect our Czech heritage with our family.}

\prompt{Write a professional email complaining about the lack of parking spaces around our office building}

These emails may contain details that are not true?   They may be over long?  What do you think?


\myslide{Summarize}

\prompt{Summarize "the War with the Newts"}

\prompt{Summarize "the War with the Newts" by Karel Čapek}

\prompt{Summarize "the War with the Newts" by Conan Doyle}

\prompt{Summarize "the War with the Newts" from the newts' point of view}

Beware of hallucinations!

\myslide{Code}

\prompt{Make a website using flask for the computational linguistics group with pages for teaching, research and news}

\prompt{Add a navigation bar}


\myslide{Debug}

\prompt{
class Test:\\
\makebox[2em]{}def printToTen(self):\\
\makebox[4em]{}ten = 10\\
\makebox[4em]{}print(ten)\\
\\
test = Test()\\
test.printToTen() \# Should print 10\\


Why does my Python code not iterate from 1 to 10 and print all numbers out?
}


\myslide{Key Principles of Prompt Engineering}
\begin{itemize}
 \item{Clarity and Specificity in Prompts}
  \begin{itemize}
  \item Avoiding Ambiguity
  \item Defining Clear Context and Goals
  \end{itemize}
  
\item{Tailoring Prompts to Model Capabilities}
  \begin{itemize}
  \item Understanding Model Strengths and Limitations
  \item Crafting Prompts that Align with Model Capabilities
  \end{itemize}
  
\item{Iterative Refinement}
  \begin{itemize}
  \item Experimentation with Various Prompts
  \item Analyzing Model Responses for Improvement
  \end{itemize}
\end{itemize}
  

\section{Discussion}

\myslide{LLM pros and cons}

  \begin{itemize}
  \item Mainly grammatical text
    \begin{itemize}
    \item For English, although no one knows if it is the same as
      human produced text %\hfill  \txx{We should study this: with DELPH-IN, GWA}
    \end{itemize}
  \item Good at transformation
    \begin{itemize}
    \item Translation
    \item Text to code (quicker but less secure)
    \item Text to image
    \end{itemize}
  \item Very expensive to train
    \begin{itemize}
    \item Only corporations can build the latest models
    \item Uses a lot of computing power
    \item Big carbon footprint
      \\ every month it uses (estimated) 4 M KWh $\approx$ as much as 30 Danes
      \\ to handle  $\approx$ 3,000 million requests!
    \end{itemize}
%%%%% https://towardsdatascience.com/chatgpts-electricity-consumption-7873483feac4
    \newpage
  \item It learns from the web
    \begin{itemize}
    \item A very wide range of knowledge
    \item[\Bad] Including sexist, racist and other toxic language
    \end{itemize}
  \item It does not have a model of the world
    \begin{itemize}
    \item GPT-3 models relationships between words without having an understanding of the meaning behind the utterances
    \item So it makes stuff up (\txx{hallucinates})
      \\ and there is no way of knowing when it is doing so
    \item Fixing this is a hot-topic of research, but there is no clear answer
    \end{itemize}
  \item Some of its training data is copy-righted text
    \begin{itemize}
    \item It is not clear whether this is \txx{legal} (probably) or \\ \emp{ethical}
      (probably not if it can closely reproduce the input text/image/code)
    \item This is especially a problem for visual models
    \end{itemize}
    \newpage
  \item Text produced by a language model is always high-probability text
    \begin{itemize}
    \item So it tends to be blah
    \item The programming is not-elegant
    \item The pictures are derivative
    \end{itemize}
    We may end up being drowned in mediocre AI produced crap
    \begin{itemize}
    \item  SF magazines are already reporting an increase in submissions
    \item \emp{If students can use it to save time, then they will}
    \item \txx{We have to set problems that cannot just be solved using LLMs}
    \item \txx{We should teach students to use it effectively \\
        --- like a calculator for text}
    \end{itemize}
    \bigskip
    \item LLMs may become self-aware and super-intelligent and kill all humans
    \begin{itemize}
    \item Unlikely given their lack of understanding or linking to the physical world
    \item They are building some internal models
      \\ but they have no understanding or actual planning
%    \item \emp{Hinton just resigned from Google} 2023-05-01!
    \end{itemize}
  \end{itemize}

% \begin{frame}{Possible Research Directions}
%   \begin{itemize}
%   \item Use our current linguistic research to see how close LLM
%     generated text is to human text
%     \begin{itemize}
%     \item We can look at syntax
%     \item We can look at word meanings/translations
%     \end{itemize}
%     Interesting but only marginally, need to rerun frequently
%   \item Experiment with using LLMs to automate some linguistic tasks
%     \begin{itemize}
%     \item Creating Lexicons
%       \begin{itemize}
%       \item I did a small pilot, good for common words in major
%         languages, unreliable for rare words or smaller languages
%       \item Size matters: Chat GPT-3 much, much better than smaller
%         models like Llama or Alpaca
%       \end{itemize}
%     \item I would like to try it for Word Sense Disambiguation
%     \end{itemize}
 
%   \end{itemize}
% \end{frame}
  
% \myslide{Risks and Ethical Issues}
%   \begin{itemize}
%   \item It will be harder to get funding to do non-neural net
%     computational linguistic research
%     \\ or linguistics or writing or translation, \ldots
%   \item But it still needs to be done
%   \item Building LLMs is fundamentally an engineering problem
%     \begin{itemize}
%     \item Understanding them is the issue (but requires strong
%       computational skill)
%     \end{itemize}
%   \item Copyright and ownership issues are up in the air
%   \item We need to have a coherent policy on students using LLMs
%     \begin{itemize}
%     \item Make sure the students have a level playing field
%     \item Make sure they don't use them when they need to practice
%       basic skills
%     \item But allow them to use them when appropriate?
%     \item Similar issue to using machine translation in language classes.
%     \end{itemize}
%   \item I think the research we will do should not pose an existential risk
%   \end{itemize}












\myslide{ChatGPT}

\myslide{If ChatGPT Can Do It, It’s Not Worth Doing}

  \href{https://www.insidehighered.com/opinion/blogs/just-visiting/2023/09/20/chatgpt-shows-way-toward-our-own-humanity}{If ChatGPT Can Do It, It’s Not Worth Doing}
\href{https://www.insidehighered.com/}{Inside Higher Ed} by  John Warner,  2023
  
\href{https://lwn.net/Articles/945504/}{AI from a legal perspective} Linux Weekly News by Jake Edge September 26, 2023

\href{https://www.wired.com/story/fast-forward-chatbot-hallucinations-are-poisoning-web-search/}{Chatbot Hallucinations Are Poisoning Web Search} Wired, ILL KNIGHTBUSINESSOCT 5, 2023 12:00 PM (accessed 2023-10-06)
  

\href{https://retractionwatch.com/2023/10/06/signs-of-undeclared-chatgpt-use-in-papers-mounting/}{Signs of undeclared ChatGPT use in papers mounting} \textit{Retraction Watch} October 6, 2023 Frederik Joelving (accessed 2023-10-03)



\href{https://annals-csis.org/proceedings/2023/pliks/3725.pdf}{An Evaluation of a Zero-Shot Approach to
Aspect-Based Sentiment Classification in Historic
German Stock Market Reports} (2023) Janos Borst, Lino Wehrheim, Andreas Niekler, Manuel Burghardt
Preprints of Communication Papers of the of the 18th Conference on Computer
Science and Intelligence Systems pp. 51–60

\href{https://blog.mathieuacher.com/GPTsChessEloRatingLegalMoves/}{Debunking the Chessboard: Confronting GPTs Against Chess Engines to Estimate Elo Ratings and Assess Legal Move Abilities} 
Mathieu Acher Blog, September 30, 2023, accessed 2023-10-18

\href{https://bdtechtalks.com/2023/05/17/llm-emergent-abilities-mirage/}{Are the emergent abilities of LLMs like GPT-4 a mirage?} TechTalks
By Ben Dickson -May 17, 2023, accessed 2023-10-19

\href{https://pluralistic.net/2023/11/27/10-types-of-people/}{The real AI fight} Pluralistic, by Cory Doctorow (27 Nov 2023)

%%
%% Good source for a tutorial:
\myslide{Acknowledmegments}

\begin{itemize}
\item Some prompt examples taken from
  \href{https://www.w3schools.com/gen_ai/chatgpt-3-5/index.php}{ChatGPT-3.5
    Tutorial} from W3 Schools
\item \url{https://rollbar.com/blog/how-to-debug-code-using-chatgpt/}
\end{itemize}
%% 


%I have taken slides from many of these.

\end{document}

%%% Local Variables: 
%%% coding: utf-8
%%% mode: latex
%%% TeX-PDF-mode: t
%%% TeX-engine: xetex
%%% End: 
